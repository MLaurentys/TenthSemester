\documentclass[]{article}
\usepackage{amsmath}
\usepackage{bm}
\usepackage{amssymb}
\usepackage{listings}
\usepackage{mathtools}
\usepackage{siunitx}

\DeclarePairedDelimiter\floor{\lfloor}{\rfloor}

\lstset{columns=fullflexible,
	mathescape=true,
	numbers=left,
	stepnumber=1,
	literate=
	{=}{$\leftarrow{}$}{1},
	morekeywords={se,entao,para,devolva,continua},
	xleftmargin=5.0ex
}

\newcommand{\binRepr}[3]{\fbox{ #1 }\fbox{ \num[group-minimum-digits = 50, minimum-integer-digits = 8]{#2} }\fbox{ \num[group-minimum-digits = 50,minimum-integer-digits = 23]{#3} }}

\title{\vspace{-4.0cm}MAC0300 - Lista 1}
\author{Matheus T. de Laurentys, 9793714}

\begin{document}
	\maketitle
	\noindent
	\textbf{PROMPT: } Exercises 4.1, 5.1, 5.8 from IEEE (Overton) book.
	
	\noindent
	\textbf{4.1} Give the single precision float representation of the following numbers: $2, 30, 31, 32, 33, 23/4, (23/4)\times 2^{100}, (23/4)\times 2^{-100}, (23/4)\times 2^{-135}, 1/5, 1024/5, \mathmbox{(1/10)\times 2^{-140}}$
	
	(2) $\longrightarrow$ \binRepr{0}{10000000}{0}\\
	
	(30) $\longrightarrow$ \binRepr{0}{10000011}{11100000000000000000000}\\
	
	(31) $\longrightarrow$ \binRepr{0}{10000011}{11110000000000000000000}\\
	
	(32) $\longrightarrow$ \binRepr{0}{10000100}{00000000000000000000000}\\
	
	(33) $\longrightarrow$ \binRepr{0}{10000100}{00001000000000000000000}\\
	
	(23/4) $\longrightarrow$ \binRepr{0}{10000001}{01110000000000000000000}\\
	
	($(23/4)\times 2^{100}$) $\longrightarrow$ \binRepr{0}{11100101}{01110000000000000000000}\\
	
	($(23/4)\times 2^{-100}$) $\longrightarrow$ \binRepr{0}{00011101}{01110000000000000000000}\\
	
	($(23/4)\times 2^{-135}$) = $(1.4375 \times 2^{-7}) \times 2^{-126}$ $\longrightarrow$\binRepr{0}{0}{10111000000000000}\\
	
	($1/5 = 1/8 \times 8/5$) $\longrightarrow$ \binRepr{0}{1111100}{10011001100110011001101}\\
	
	($1024/5$) $\longrightarrow$ \binRepr{0}{10000110}{10011001100110011001101}\\
	
	($(1/10)\times 2^{100}$) $\longrightarrow$ \binRepr{0}{11011111}{10011001100110011001101}\\
	
	
	\noindent
	\textbf{5.1} Give the rounded values of 1/10, using each of the rounding modes? What are they for ($1+2^{-25}$) and $2^{130}$.
	
	(1/10):\\
	Round Down: \binRepr{0}{01111011}{10011001100110011001110}\\
	Round Up: \binRepr{0}{01111011}{10011001100110011001101}\\
	Round Towards Zero: \binRepr{0}{01111011}{10011001100110011001110}\\
	Round to Nearest: \binRepr{0}{01111011}{10011001100110011001101}\\
	
	($1+2^{-25}$)\\
	Round Down: \binRepr{0}{1111111}{0}\\
	Round Up: \binRepr{0}{1111111}{1}\\
	Round Towards Zero: \binRepr{0}{1111111}{0}\\
	Round to Nearest: \binRepr{0}{1111111}{0}\\
	
	\pagebreak
	($2^{130}$)\\
	Round Down: \binRepr{0}{11111110}{11111111111111111111111}\\
	Round Up: \binRepr{0}{11111111}{0}\\
	Round Towards Zero: \binRepr{0}{11111110}{11111111111111111111111}\\
	Round to Nearest: \binRepr{0}{11111111}{0}\\
	
	\noindent \textbf{5.8} Do bounds (5.10) and (5.11) hold when $|x| < N_{min}$?\\
	
	No, the relative rounding error bounds do not hold, but as seen in exercise 5.7, the absolute rounding error bound holds.
	
	For example, let $0 < x < (2^{-23} \times 2^{-126})$.
	In this case, either round(x) = 0, or round(x) = $2^{-23} \times 2^{-126}$. In either case it is easy to verify that the rounding error is greater than $\epsilon$ since the round to nearest error can be up to $2^{-24} \times 2^{-126}$ and, in this case, the relerr is $\frac{1}{2}$.
\end{document}


















