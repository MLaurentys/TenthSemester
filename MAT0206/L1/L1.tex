\documentclass[]{article}
\usepackage{amsmath}
\usepackage{bm}
\usepackage{amssymb}
\usepackage{listings}
\usepackage{mathtools}
\usepackage{siunitx}
\usepackage{adjustbox}
\usepackage{setspace}

\DeclarePairedDelimiter\floor{\lfloor}{\rfloor}

\lstset{columns=fullflexible,
	mathescape=true,
	numbers=left,
	stepnumber=1,
	literate=
	{=}{$\leftarrow{}$}{1},
	morekeywords={se,entao,para,devolva,continua},
	xleftmargin=5.0ex
}

\newcommand{\binRepr}[3]{\fbox{ #1 }\fbox{ \num[group-minimum-digits = 50, minimum-integer-digits = 8]{#2} }\fbox{ \num[group-minimum-digits = 50,minimum-integer-digits = 23]{#3} }}
\newcommand{\marginVBox}[1]{\adjustbox{margin*=1cm 0cm 0cm 0cm}{\vbox{\setstretch{1.6}#1}}}

\title{\vspace{-4.0cm}MAT0206 - Lista 1}
\author{Matheus T. de Laurentys, 9793714}

\begin{document}
	
	\maketitle
	
	\noindent \textbf{1.} $A \subseteq B \iff A \cap B = A \iff A \cup B = B$\\
	
	Proving that: $A \subseteq B \implies A \cap B = A$:\\
	
	\marginVBox {
		\noindent $(x \in A \cap B \rightarrow x \in A) \rightarrow A \cap B \subseteq A$ \\
		$A \subseteq B \rightarrow (x \in A \rightarrow x \in B) \rightarrow A \subseteq A \cap B$ \\
		$(A \cap B \subseteq A) \land (A \subseteq A \cap B) \rightarrow A = A \cap B$
	}\\

	Proving that: $A \cap B = A \implies A \subseteq B$:\\
	
	\marginVBox {
		\noindent $A \cap B = A \rightarrow (x \in A \rightarrow x \in A \cap B \rightarrow x \in B) \rightarrow A \subseteq B$
	}\\

	Proving that: $A \cap B = A \implies A \cup B = B$:\\
	
	\marginVBox{
		\noindent 
	}
	\end{document}

























