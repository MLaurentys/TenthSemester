\documentclass[15pt]{article}
\usepackage[a4paper,width=130mm,headheight=110pt,top=3cm,bottom=25mm]{geometry}

\usepackage[utf8]{inputenc}
\usepackage[utf8]{inputenc}
\usepackage{listings}
\usepackage{graphicx}
\usepackage{mathdots}
\usepackage{tikz}
\usepackage{pgffor} 
\usepackage{float}
\usepackage{graphics} 
\usepackage{fancyhdr}
\usepackage[square, sort, numbers]{natbib}
\usepackage{color}
\usepackage{indentfirst}
\usepackage{epigraph}
\usepackage{ragged2e}
\usepackage{blindtext}
\usepackage{amsmath,amsthm,amssymb}
\usepackage{tabto}
\usepackage{pgfplots}
\usepackage{changepage}
\usepackage{subcaption}
\usepackage{fancyvrb}
\usepackage{caption}
\usepackage{mathtools}
\usepackage{setspace}
\doublespacing

\title{MAT0206 - Prova 3}
\author{Matheus T. de Laurentys, 9793714}

\newcommand{\f}{\mkern-2mu f\mkern-3mu}

\newcommand{\acumulacao}{acumulação }
\newcommand{\adh}{aderência }
\newcommand{\arbitrario}{arbitrário }
\newcommand{\continua}{contínua }
\newcommand{\contradicao}{contradição }
\newcommand{\comparacao}{comparação }
\newcommand{\criterio}{critério }
\newcommand{\definicao}{definição }
\newcommand{\derivavel}{derivável }
\newcommand{\diferenciavel}{diferenciável }
\newcommand{\dominio}{domínio }
\newcommand{\entao}{então }
\newcommand{\eh}{é }
\newcommand{\Eh}{É }
\newcommand{\Entao}{Então }
\newcommand{\enumeravel}{enumerável }
\newcommand{\funcao}{função }
\newcommand{\inequacao}{inequação }
\newcommand{\nao}{não }
\newcommand{\num}{número }
\newcommand{\particao}{partição }
\newcommand{\raizes}{raízes }
\newcommand{\razao}{razão }
\newcommand{\sequencia}{sequência }
\newcommand{\sequencias}{sequências }
\newcommand{\sao}{são }
\newcommand{\serie}{série }
\newcommand{\subtracao}{subtração }
\newcommand{\tambem}{também }
\newcommand{\vizin}{vizinhança }
\newcommand{\variacao}{variação }

\newcommand{\seqn}[1]{\mbox{$(#1_n)_{n\in\N}$}}
\newcommand{\limi}[2]{\mbox{$\displaystyle{\lim_{n \to +\infty}}(#1)=#2$}}
\renewcommand{\liminf}[1]{\mbox{$\underset{#1}{\text{lim inf}}$}}
\renewcommand{\limsup}[1]{\mbox{\underset{$#1$}{\text{lim sup}}}}
\renewcommand{\f}{\mkern-2mu f\mkern-3mu}
\newcommand{\R}{\mathbb{R}}
\newcommand{\N}{\mathbb{N}}
\newcommand{\series}[1]{\mbox{$\sum\limits_{i \in #1}$}}

\DeclarePairedDelimiter\ceil{\lceil}{\rceil}
\DeclarePairedDelimiter\floor{\lfloor}{\rfloor}

\begin{document}
	\maketitle
	
	\noindent\textbf{Q.1}

\textbf{a)}

Seja $I \subset \N$ o conjunto dos indices de \seqn{x} tais que $\sqrt{x_n} \leq \frac{1}{n}$.

Tomando $n \in \N$, se $n\in I$, \entao $\frac{\sqrt{x_n}}{n} \leq \frac{1}{n^2}$, pois $\sqrt{x_n} \leq \frac{1}{n}$. Caso $n \notin I$, \entao $\frac{\sqrt{x_n}}{n} < \sqrt{x_n}^2 = x_n$. Sabemos \entao que:

$$\series{\N}\frac{\sqrt{x_i}}{i} = \series{\N \setminus I}\frac{\sqrt{x_i}}{i} + \series{I}\frac{\sqrt{x_i}}{i} \leq \series{I}\frac{1}{i^2} + \series{\N\setminus I}x_i$$

\Eh dado que $\series{\N}x_i$ converge. A \serie $\series{\N}\frac{1}{i^2}$ \eh uma PG com \razao menor que 1 e, por isso, converge (ja foi visto em aula). A soma de \serie s convergentes \tambem converge. Assim, tome \series{\N} $(x_i + \frac{1}{i^2})$. Notando que \series{\N} $\frac{\sqrt{x_i}}{i} \leq$ \series{\N} $(x_i + \frac{1}{i^2})$, podemos afirmar que $\frac{\sqrt{x_i}}{i}$ converge pelo \criterio de \comparacao $(i)$.

\textbf{b)}

Seja \liminf{n\in\N} $y_n = \alpha > 0$. Tome $\epsilon$ tal que $\alpha - \epsilon > 0$ e seja $\beta = \alpha - \epsilon$. Existe apenas um numero finito de elementos da \sequencia \seqn{y} menores que $\beta$ pela \definicao de lim inf. Seja $I$ o conjunto dos indices $n$ tais que $y_n > \beta$. 

Se $\beta \ge 1$, \entao $\forall i \in I, \frac{x_i}{y_i} \leq x_i$. Sendo assim, pelo \criterio da \comparacao ($i$), a \serie \series{I} $\frac{x_i}{y_i}$ converge. Como \series{\N \setminus I} $\frac{x_i}{y_i}$ \eh soma finita, \entao \series{\N} $\frac{x_i}{y_i}$ converge.

Se $\beta < 1$, \entao $\forall i \in I, \frac{x_1}{y_i} \leq \frac{x_1}{\beta}$. Portanto, tem-se que:

$$\series{I} \frac{x_i}{y_i} \leq \series{I} \frac{x_i}{\beta} = \frac{1}{\beta} \series{I} x_n \leq \frac{1}{\beta} \series{\N} x_n$$

Como $\series{\N} x_n$ converge, \entao $\series{I} \frac{x_i}{y_i}$ \tambem converge. Novamente, como \series{\N \setminus I} $\frac{x_i}{y_i}$ \eh soma finita, \entao \series{\N} $\frac{x_i}{y_i}$ converge.
	
	\textbf{Q.2}

\textbf{a)}

Tome $h:X\rightarrow\R$ dada por $h = f\cdot g$. Dado $\epsilon > 0$.

Como $f$ e $g$ \sao limitadas, temos que $\exists L\in\R; \forall x\in X; L>|f(x)|$ e $L>|g(x)|$. Fixe qualquer tal $L$ e tome \tambem $\beta = \frac{\epsilon}{L}$.


Como $f$ e $g$ uniformemente \continua s temos que $\exists \delta_f,\delta_g > 0$ tais que \mbox{$\forall x,y \in X$}, $|x-y| < \delta_f \Rightarrow |f(x)-f(y)| < \frac{\beta}{2}$ e $|x-y| < \delta_g \Rightarrow |g(x)-g(y)| < \frac{\beta}{2}$.


Fixe \entao $\delta = \min\{\delta_f, \delta_g\}$. Tomando $x,y \in X$ com $|x-y| < \delta$, temos:

\begin{align*}
	|h(x) - h(y)| &= |f(x)\cdot g(x) - f(y)\cdot g(y)| \\
	&= |f(x)\cdot g(x) + f(x)\cdot g(y) - f(y)\cdot g(y) - f(x)\cdot g(y)| \\
	&= |f(x)\cdot(g(x) + g(y)) - g(y)\cdot (f(x) + f(y))| \\
	&\leq ||f(x)\cdot(g(x) + g(y))| + |g(y)\cdot (f(x) + f(y))||\\
	&= ||f(x)|\cdot|(g(x) + g(y))| + |g(y)|\cdot|(f(x) + f(y))||
\end{align*}

Usando a hipotese de que $f$ e $g$ \sao limitadas e $L>|f(x)|$ e $L>|g(x)|$, tem-se:

\begin{align*}
	|h(x) - h(y)| &\le ||f(x)|\cdot|(g(x) + g(y))| + |g(y)|\cdot|(f(x) + f(y))|| \\
	&< |L\cdot|(g(x) + g(y))| + L\cdot|(f(x) + f(y))||\\
	&= L(||(g(x) + g(y))| + |(f(x) + f(y))||)
\end{align*}

Usando agora que $|f(x)-f(y)| < \frac{\beta}{2}$ e $|g(x)-g(y)| < \frac{\beta}{2}$ (devido a escolha de $x,y$), temos:

\begin{align*}
	|h(x) - h(y)| &< L(||(g(x) + g(y))| + |(f(x) + f(y))||)\\
	&= L(|(g(x) + g(y))| + |(f(x) + f(y))|)\\
	&\le L(\frac{\beta}{2} + \frac{\beta}{2})
\end{align*}

Por fim,

\begin{align*}
	|h(x) - h(y)| &<  L(\frac{\beta}{2} + \frac{\beta}{2}) \\
	&= L\cdot\frac{\epsilon}{L}\\
	&=\epsilon
\end{align*}

Sendo assim, dado qualquer $\epsilon > 0$, existe $\delta > 0$ tal que \mbox{$|x-y| < \delta \Rightarrow |h(x)-h(y)|<\epsilon$}. Isso mostra que a \funcao $h = f\cdot g$ \eh uniformemente \continua.

\textbf{b}

Tome $f,g:\R\rightarrow\R$ ambas dadas por $f(x) = g(x) = x$. Tome $h = f\cdot g$.

Tanto $f$ quanto $g$ \sao uniformemente \continua s.

Prova: Dado $\epsilon > 0$, tome $\delta = \epsilon$. Assim, temos que $|x - y| < \delta \Rightarrow |f(x) - f(y)| < \epsilon$, pois $|f(x) - f(y)| = |x - y| < \delta = \epsilon$.

No entanto, $h = f \cdot g$ dada por $h(x) = x^2$ \nao \eh uniformemente \continua.

Prova [por \contradicao]:

Tome $\epsilon$ qualquer. Como $h$ uniformemente \continua temos que $\exists \delta > 0, \forall x,y \in \R, |x-y| < \delta \Rightarrow |f(x)-f(y)| < \epsilon$. Tome qualquer tal $\delta$.

Tome $y = x + \frac{\delta}{2}$, temos que $|x-y| = \frac{\delta}{2} < \delta$. Assim, 

\begin{align*}
	|f(x) - f(y)| &= |x^2 - x^2 - x\delta - \frac{\delta^2}{4}| \\
	&= |- x\delta - \frac{\delta^2}{4}| \\
	&= |x\delta + \frac{\delta^2}{4}| \\
	&\ge |x\delta| - |\frac{\delta^2}{4}| \\
	&= |x|\delta - \frac{\delta^2}{4}
\end{align*}

No entando, se $|x| \ge  \frac{\epsilon + \frac{\delta^2}{4}}{\delta}$, temos \contradicao visto que:
\begin{align*}
	|f(x) - f(y)| &\ge |x|\delta - \frac{\delta^2}{4} \\
	&\ge \epsilon + \frac{\delta^2}{4} - \frac{\delta^2}{4} \\
	&= \epsilon
\end{align*}

Vale ressaltar que tal escolha de $x$ \eh sempre possivel vide o dominio.




	
	\textbf{Q.3}

Note que como a \funcao $f$ \eh \continua no intervalo $[a,b]$, ela \tambem \eh limitada. Sendo assim, ela \eh integravel.

%\Eh fato \tambem que $|f|$ \eh integravel. pois se
%
%$$
%\underset{P}{\sup}\{\sum_{i=1}^n \inf (f([t_{i-1}, t_i]))\cdot(t_i - t_{i-1})\} = \underset{P}{\inf}\{\sum_{i=1}^n \sup (f([t_{i-1}, t_i]))\cdot(t_i - t_{i-1})\}
%$$

Seja $\epsilon = \frac{f(x_0)}{2}$. Como $f$ \continua, $\exists \delta > 0$ tal que $\forall x \in (x_0 - \delta, x_0 + \delta)\cap [a,b]$ tem-se que $f(x) \in (f(x_0) - \epsilon, f(x_0) + \epsilon)$. Tome uma \particao $P_* = \{a=x_0 < \ldots < x_n=b\}$ com um dos intervalos $I_k = [x_{k-1}, x_k]$ tal que $I_k \subset (x_0 - \delta, x_0 + \delta)\cap [a,b]$.

\begin{align*}
	\int_a^b|f(x)|dx &= \underset{P}{\sup}\{s(|f|;P); P \text{ \eh \particao de } [a,b]\} \\
	&\ge s(|f|;P_*) \\
	&= \sum_{i=1}^n \inf (|f([x_{i-1}, x_i])|)\cdot(x_i - x_{i-1}) \\
	&\ge \inf (|f([x_{k-1}, x_k])|)\cdot |[x_{k-1}, x_k]| \\
	&\ge |(f(x_0) - \epsilon)|\cdot |[x_{k-1}, x_k]| \text{ * }\\
	&= |\frac{f(x_0)}{2}|\cdot |[x_{k-1}, x_k]|
\end{align*}

* $\forall x \in [x_{k-1}, x_k]$ tem-se $f(x) \in (f(x_0) - \epsilon, f(x_0) + \epsilon)$, devido a escolha de $P$ mediante $I_k \subset (x_0 - \delta, x_0 + \delta)\cap [a,b]$.

Como $f(x_0) \neq 0, |\frac{f(x_0)}{2}| > 0$. Notando que $[x_{k-1}, x_k] = (x_0 - \delta, x_0 + \delta)\cap [a,b] = I_i$ \eh um intervalo, temos que $|[x_{k-1}, x_k]| = x_k - x_{k-1}$, por \definicao. Alem disso, $x_k - x_{k-1} > 0$, pois, novamente, $I_k$ \eh intervalo. Dessa forma,

$$
\int_a^b|f(x)|dx \ge |\frac{f(x_0)}{2}|\cdot |[x_{k-1}, x_k]| > 0
$$

	
	\textbf{Q.4} Mostar que, dado $X \subset \R$.

$\forall \epsilon > 0$ existe cobertura $I$ enumeravel de $X$ por intervalos abertos tal que \series{\N} $|I_i| < \epsilon \iff$ $\forall \epsilon > 0$ existe cobertura $J$ enumeravel de $X$ por intervalos fechados tal que \series{\N} $|J_i| < \epsilon$.

($\Rightarrow$)

Dado $\epsilon > 0$. Seja $I$ uma cobertura enumeravel por abertos de $X$ tal que \mbox{\series{\N}~$|I_i| < \epsilon$}. Considere agora um conjunto enumeravel de intervalos fechados $J$ dado por $J_n~=~I_n~+~\partial~I_n$, isto \eh, se $I_n = (a,b)$, \entao $J_n = [a,b]$. Notando que qualquer $x\in X$ coberto por algum $I_n \in I$ \eh \tambem coberto por $J_n$, \entao $J$ \eh uma cobertura enumeravel de $X$. Tambem \eh verdade que $\forall n \in \N, |J_n| = |I_n|$, por \definicao. Sendo assim, \series{\N} $|J_i|$ = \series{\N} $|I_i| < \epsilon$.

Temos \entao que $\forall \epsilon > 0$ existe cobertura $I$ enumeravel de $X$ por intervalos abertos tal que \series{\N} $|I_i| < \epsilon \Rightarrow$ $\forall \epsilon > 0$ existe cobertura $J$ enumeravel de $X$ por intervalos fechados tal que \series{\N} $|J_i| < \epsilon$.

($\Leftarrow$)

Dado $\epsilon > 0$, tome $\delta$ tal que $\delta < \frac{\epsilon}{2}$. Seja $J$ uma cobertura enumeravel por fechados de $X$ tal que \series{\N}~$|J_i| < \delta$. Seja $L$ um segundo conjunto enumeralvel de intervalos fechados criado a partir de $J$. $L$ \eh tal que, se $J_n = [a,b]$, \entao $L_n = [a-\frac{b-a}{2}, b + \frac{b-a}{2}]$. \Eh claro que $L$ \tambem \eh uma cobertura por fechados de $X$ pois $\forall n \in \N, J_n \subset L_n$ e $\forall x \in X, \exists n \in \N$ tal que $x \in J_n, x \in L_n$.

Tambem \eh claro que $\forall n \in \N, |L_n| = 2\times |J_n|$, pela forma como $L_n$ foi definido. Sendo assim, 

$$\series{\N} |L_i| = 2\times\series{\N} |J_i| < 2\times \delta < \epsilon$$

Considere, agora, o conjunto \enumeravel de intervalos abertos $I$ tal que, $\forall n \in \N$, se $L_n = [a,b]$ \entao $I_n = (a,b)$. Notando que, $\forall n \in \N, J_n \subset I_n$ (pela forma como $L$ foi contruido), $I$ \tambem \eh uma cobertura de $X$. Novamente, como $\forall n \in \N, |I_n| = |L_n|$, \entao \series{\N}$|I_i| = \series{\N}|L_i| < \epsilon$. Logo, $I$ \eh uma cobertura \enumeravel de $X$ e \series{\N}$|I_i| < \epsilon$.

Temos \entao que $\forall \epsilon > 0$ existe cobertura $J$ \enumeravel de $X$ por intervalos fechados tal que \series{\N} $|J_i| < \epsilon \Rightarrow$ $\forall \epsilon > 0$ existe cobertura $I$ \enumeravel de $X$ por intervalos abertos tal que \series{\N} $|I_i| < \epsilon$.

	
	\textbf{Q.5}

\textbf{a)}

\begin{align*}
	f_+'(a) = \lim_{x\rightarrow a_+} \frac{f(x)-f(a)}{x-a} < 0
\end{align*}

Tome $x\in X\cap(a,a+\delta)$.

\begin{align*}
	f(x) - f(a) &= \frac{f(x)-f(a)}{x-a}(x-a) \\
	&\Longrightarrow \\
	\lim_{x\rightarrow a_+} f(x)-f(a) &= \lim_{x\rightarrow a_+} \frac{f(x)-f(a)}{x-a}(x-a) \\
	&= \lim_{x\rightarrow a_+} \frac{f(x)-f(a)}{x-a} \cdot \lim_{x\rightarrow a_+}(x-a) \\
	&= f_+'(a) \cdot 0\\
	&= 0
\end{align*}

Isso mostra que a \funcao $f$ \eh \continua pela direita em $x=a$. Sendo assim, dado $\epsilon >0, \exists \delta > 0$ tal que se $x\in X \cap (a, a+ \delta)$, \entao $f(x) \in (f(a) - \epsilon, f(a) + \epsilon)$. 

Como $\lim_{x\rightarrow a_+} \frac{f(x)-f(a)}{x-a} = L < 0$, tome $\epsilon = \frac{L}{2}$. Como visto acima, $\exists x_0 > a$ tal que se $x \in (a, x_0)$ tem-se que $\frac{f(x)-f(a)}{x-a} \in (L -\epsilon, L + \epsilon)$. Tomando $\delta = x_0 - a$.

Tome $x \in (a, a + \delta)$ qualquer. Como $\frac{f(x)-f(a)}{x-a} \in (L -\epsilon, L + \epsilon)$, temos que  $\frac{f(x)-f(a)}{x-a} < 0$. Sendo assim e como $x > a \rightarrow x-a>0$, temos $f(x) - f(a) > 0$. 

Finalmente \entao, $f(x) > f(a), \forall x \in (a, a + \delta)$.

\textbf{b)}

Seja $a\in X\cap X_-'$ e $f:X\rightarrow\R$ \derivavel a esquerda em $a$. Mostre que se $f_-'(a) < 0$ \entao existe $\delta > 0$ tal que se $x \in X (a-\delta, a)$ \entao $f(a) > f(x)$.

\begin{align*}
	f_-'(a) = \lim_{x\rightarrow a_-} \frac{f(x)-f(a)}{x-a} < 0
\end{align*}

Tome $x\in X\cap(a - \delta,a)$.

\begin{align*}
	f(x) - f(a) &= \frac{f(x)-f(a)}{x-a}(x-a) \\
	&\Longrightarrow \\
	\lim_{x\rightarrow a_-} f(x)-f(a) &= \lim_{x\rightarrow a_-} \frac{f(x)-f(a)}{x-a}(x-a) \\
	&= \lim_{x\rightarrow a_-} \frac{f(x)-f(a)}{x-a} \cdot \lim_{x\rightarrow a_-}(x-a) \\
	&= f_-'(a) \cdot 0\\
	&= 0
\end{align*}

Isso mostra que a \funcao $f$ \eh \continua pela esquerda em $x=a$. Sendo assim, dado $\epsilon > 0, \exists \delta > 0$ tal que se $x\in X \cap (a - \delta, a)$, \entao $f(x) \in (f(a) - \epsilon, f(a) + \epsilon)$. 

Como $\lim_{x\rightarrow a_+} \frac{f(x)-f(a)}{x-a} = L < 0$, tome $\epsilon = \frac{L}{2}$. Como visto acima, $\exists x_0 < a$ tal que se $x \in (x_0, a)$ tem-se que $\frac{f(x)-f(a)}{x-a} \in (L -\epsilon, L + \epsilon)$. Tomando $\delta = a - x_0$.

Tome $x \in (a - \delta, a)$ qualquer. Como $\frac{f(x)-f(a)}{x-a} \in (L -\epsilon, L + \epsilon)$, temos que  $\frac{f(x)-f(a)}{x-a} < 0$. Sendo assim e como $x > a \rightarrow x-a>0$, temos $f(x) - f(a) > 0$. 

Finalmente \entao, $f(x) > f(a), \forall x \in (a - \delta, a)$.






 












	
\end{document}














