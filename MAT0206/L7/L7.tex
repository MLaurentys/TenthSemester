\documentclass[15pt]{article}
\usepackage[a4paper,width=130mm,headheight=110pt,top=3cm,bottom=25mm]{geometry}

\usepackage[utf8]{inputenc}
\usepackage[utf8]{inputenc}
\usepackage{listings}
\usepackage{graphicx}
\usepackage{mathdots}
\usepackage{tikz}
\usepackage{pgffor} 
\usepackage{float}
\usepackage{graphics} 
\usepackage{fancyhdr}
\usepackage[square, sort, numbers]{natbib}
\usepackage{color}
\usepackage{indentfirst}
\usepackage{epigraph}
\usepackage{ragged2e}
\usepackage{blindtext}
\usepackage{amsmath,amsthm,amssymb}
\usepackage{tabto}
\usepackage{pgfplots}
\usepackage{changepage}
\usepackage{subcaption}
\usepackage{fancyvrb}
\usepackage{caption}
\usepackage{mathtools}
\usepackage{setspace}
\doublespacing

\title{\vspace{-4.0cm}MAT0206 - Lista 7}
\author{Matheus T. de Laurentys, 9793714}

\newcommand{\f}{\mkern-2mu f\mkern-3mu}

\newcommand{\acc}{acumulação }
\newcommand{\adh}{aderência }
\newcommand{\arbitrario}{arbitrário }
\newcommand{\continua}{contínua }
\newcommand{\contradicao}{contradição }
\newcommand{\comparacao}{comparação }
\newcommand{\criterio}{critério }
\newcommand{\definicao}{definição }
\newcommand{\derivavel}{derivável }
\newcommand{\diferenciavel}{diferenciável }
\newcommand{\dominio}{domínio }
\newcommand{\entao}{então }
\newcommand{\eh}{é }
\newcommand{\Eh}{É }
\newcommand{\Entao}{Então }
\newcommand{\enumeravel}{enumerável }
\newcommand{\funcao}{função }
\newcommand{\inequacao}{inequação }
\newcommand{\nao}{não }
\newcommand{\num}{número }
\newcommand{\particao}{partição }
\newcommand{\raizes}{raízes }
\newcommand{\razao}{razão }
\newcommand{\sequencia}{sequência }
\newcommand{\sequencias}{sequências }
\newcommand{\sao}{são }
\newcommand{\serie}{série }
\newcommand{\subtracao}{subtração }
\newcommand{\tambem}{também }
\newcommand{\vizin}{vizinhança }
\newcommand{\variacao}{variação }

\newcommand{\seqn}[1]{\mbox{$(#1_n)_{n\in\N}$}}
\newcommand{\limi}[2]{$\displaystyle{\lim_{n \to +\infty}}(#1)=#2$}
\renewcommand{\liminf}[1]{\mbox{$\underset{#1}{\text{lim inf}}$}}

\renewcommand{\f}{\mkern-2mu f\mkern-3mu}
\newcommand{\R}{\mathbb{R}}
\newcommand{\N}{\mathbb{N}}
\newcommand{\series}[1]{\mbox{$\sum\limits_{i \in #1}$}}

\DeclarePairedDelimiter\ceil{\lceil}{\rceil}
\DeclarePairedDelimiter\floor{\lfloor}{\rfloor}

\begin{document}
	\maketitle
	
	Questões feitas: 1, 4, 7, 11, 12, 13, 14, 15
	
	\textbf{Provas extra que irei usar:}

\criterio da \comparacao

Faltou demonstrar que dado \seqn{x} e \seqn{y} \sequencias de numeros positivos tais que $\forall n \in \N, x_n \leq y_n$:

($i$) Se \series{\N} $y_i$ converge, \entao \series{\N} $x_i$ converge.

Considere as somas parciais $S_n = x_1 + \ldots + x_n$ e $T_n = y_1 + \ldots + y_n$. Claro que $S_n \leq T_n$.

Considere, agora, que:

\begin{align*}
	0\leq S_n = x_i + \ldots + x_n &\leq x_i + \ldots + x_n + y_{n+1} + y_{n+2} + \ldots = S_n + (T - T_n) \\
	S_n + (T - T_n) &= T + (S_n - T_n) \leq T + (0) = T
\end{align*}

Note que \seqn{S} \eh \nao decrescente e que $(T + (S_n + T_n))_{n\in\N}$ \eh \nao crescente. Sendo assim, dados quaisquer tres naturais $n,m,L$, \eh fato que $S_n, S_m \in [S_L, T + (S_L - T_L)]$ e isso mostra que a \sequencia $S_n$ \eh de cauchy. Como visto em aula, $S_n$ ser de cauchy implica que \series{\N} $S_n$ converge.

%($ii$) Se \series{\N} $x_i$ diverge, \entao \series{\N} $y_i$ diverge.
	
	\noindent\textbf{Q.1}

\textbf{a)}

Seja $I \subset \N$ o conjunto dos indices de \seqn{x} tais que $\sqrt{x_n} \leq \frac{1}{n}$.

Tomando $n \in \N$, se $n\in I$, \entao $\frac{\sqrt{x_n}}{n} \leq \frac{1}{n^2}$, pois $\sqrt{x_n} \leq \frac{1}{n}$. Caso $n \notin I$, \entao $\frac{\sqrt{x_n}}{n} < \sqrt{x_n}^2 = x_n$. Sabemos \entao que:

$$\series{\N}\frac{\sqrt{x_i}}{i} = \series{\N \setminus I}\frac{\sqrt{x_i}}{i} + \series{I}\frac{\sqrt{x_i}}{i} \leq \series{I}\frac{1}{i^2} + \series{\N\setminus I}x_i$$

\Eh dado que $\series{\N}x_i$ converge. A \serie $\series{\N}\frac{1}{i^2}$ \eh uma PG com \razao menor que 1 e, por isso, converge (ja foi visto em aula). A soma de \serie s convergentes \tambem converge. Assim, tome \series{\N} $(x_i + \frac{1}{i^2})$. Notando que \series{\N} $\frac{\sqrt{x_i}}{i} \leq$ \series{\N} $(x_i + \frac{1}{i^2})$, podemos afirmar que $\frac{\sqrt{x_i}}{i}$ converge pelo \criterio de \comparacao $(i)$.

\textbf{b)}

Seja \liminf{n\in\N} $y_n = \alpha > 0$. Tome $\epsilon$ tal que $\alpha - \epsilon > 0$ e seja $\beta = \alpha - \epsilon$. Existe apenas um numero finito de elementos da \sequencia \seqn{y} menores que $\beta$ pela \definicao de lim inf. Seja $I$ o conjunto dos indices $n$ tais que $y_n > \beta$. 

Se $\beta \ge 1$, \entao $\forall i \in I, \frac{x_i}{y_i} \leq x_i$. Sendo assim, pelo \criterio da \comparacao ($i$), a \serie \series{I} $\frac{x_i}{y_i}$ converge. Como \series{\N \setminus I} $\frac{x_i}{y_i}$ \eh soma finita, \entao \series{\N} $\frac{x_i}{y_i}$ converge.

Se $\beta < 1$, \entao $\forall i \in I, \frac{x_1}{y_i} \leq \frac{x_1}{\beta}$. Portanto, tem-se que:

$$\series{I} \frac{x_i}{y_i} \leq \series{I} \frac{x_i}{\beta} = \frac{1}{\beta} \series{I} x_n \leq \frac{1}{\beta} \series{\N} x_n$$

Como $\series{\N} x_n$ converge, \entao $\series{I} \frac{x_i}{y_i}$ \tambem converge. Novamente, como \series{\N \setminus I} $\frac{x_i}{y_i}$ \eh soma finita, \entao \series{\N} $\frac{x_i}{y_i}$ converge.
	
	\textbf{Q.4} Mostar que, dado $X \subset \R$.

$\forall \epsilon > 0$ existe cobertura $I$ enumeravel de $X$ por intervalos abertos tal que \series{\N} $|I_i| < \epsilon \iff$ $\forall \epsilon > 0$ existe cobertura $J$ enumeravel de $X$ por intervalos fechados tal que \series{\N} $|J_i| < \epsilon$.

($\Rightarrow$)

Dado $\epsilon > 0$. Seja $I$ uma cobertura enumeravel por abertos de $X$ tal que \mbox{\series{\N}~$|I_i| < \epsilon$}. Considere agora um conjunto enumeravel de intervalos fechados $J$ dado por $J_n~=~I_n~+~\partial~I_n$, isto \eh, se $I_n = (a,b)$, \entao $J_n = [a,b]$. Notando que qualquer $x\in X$ coberto por algum $I_n \in I$ \eh \tambem coberto por $J_n$, \entao $J$ \eh uma cobertura enumeravel de $X$. Tambem \eh verdade que $\forall n \in \N, |J_n| = |I_n|$, por \definicao. Sendo assim, \series{\N} $|J_i|$ = \series{\N} $|I_i| < \epsilon$.

Temos \entao que $\forall \epsilon > 0$ existe cobertura $I$ enumeravel de $X$ por intervalos abertos tal que \series{\N} $|I_i| < \epsilon \Rightarrow$ $\forall \epsilon > 0$ existe cobertura $J$ enumeravel de $X$ por intervalos fechados tal que \series{\N} $|J_i| < \epsilon$.

($\Leftarrow$)

Dado $\epsilon > 0$, tome $\delta$ tal que $\delta < \frac{\epsilon}{2}$. Seja $J$ uma cobertura enumeravel por fechados de $X$ tal que \series{\N}~$|J_i| < \delta$. Seja $L$ um segundo conjunto enumeralvel de intervalos fechados criado a partir de $J$. $L$ \eh tal que, se $J_n = [a,b]$, \entao $L_n = [a-\frac{b-a}{2}, b + \frac{b-a}{2}]$. \Eh claro que $L$ \tambem \eh uma cobertura por fechados de $X$ pois $\forall n \in \N, J_n \subset L_n$ e $\forall x \in X, \exists n \in \N$ tal que $x \in J_n, x \in L_n$.

Tambem \eh claro que $\forall n \in \N, |L_n| = 2\times |J_n|$, pela forma como $L_n$ foi definido. Sendo assim, 

$$\series{\N} |L_i| = 2\times\series{\N} |J_i| < 2\times \delta < \epsilon$$

Considere, agora, o conjunto \enumeravel de intervalos abertos $I$ tal que, $\forall n \in \N$, se $L_n = [a,b]$ \entao $I_n = (a,b)$. Notando que, $\forall n \in \N, J_n \subset I_n$ (pela forma como $L$ foi contruido), $I$ \tambem \eh uma cobertura de $X$. Novamente, como $\forall n \in \N, |I_n| = |L_n|$, \entao \series{\N}$|I_i| = \series{\N}|L_i| < \epsilon$. Logo, $I$ \eh uma cobertura \enumeravel de $X$ e \series{\N}$|I_i| < \epsilon$.

Temos \entao que $\forall \epsilon > 0$ existe cobertura $J$ \enumeravel de $X$ por intervalos fechados tal que \series{\N} $|J_i| < \epsilon \Rightarrow$ $\forall \epsilon > 0$ existe cobertura $I$ \enumeravel de $X$ por intervalos abertos tal que \series{\N} $|I_i| < \epsilon$.

	
	\textbf{Q.7}

\textbf{a)} Mostrar que 
$$\exists \epsilon > 0 \text{ tal que } \forall \delta > 0, \exists x,y \in (0,+\infty) \text{ com } |x-y| < \delta \text{ e } |f(x) - f(y)| > \epsilon$$

Fixando $\epsilon = 1$. Tomando $\delta > 0$ qualquer.

Caso $\delta \ge \frac{2}{\pi}$

Toma-se $x=\frac{1}{2}$ e $y=\frac{1}{10}$. Dessa forma, $|x-y| = |0.5-0.1| = 0.4 < \frac{2}{\pi}$, porem, $|f(x) - f(y)| = |\sin(2) - \sin(10)| \approx 1.45 > 1$.

Caso $\delta < \frac{2}{\pi}$:

Fixando $k$ algum inteiro. Tomar $x$ da forma $\frac{1}{2\times\pi\times k - \frac{\pi}{2}}$ de forma que \mbox{$\frac{1}{x} = 2\pi k - \frac{\pi}{2}$} e, assim, $\sin(\frac{1}{x}) = -1$. Tomar $y$ da forma $\frac{1}{2\times\pi\times k + \frac{\pi}{2}}$ de forma que $\frac{1}{y} = 2\pi k + \frac{\pi}{2}$ e, assim, $\sin(\frac{1}{y}) = 1$.

Para que $|x-y| < \delta$, temos:
\begin{align*}
|\frac{1}{2\pi k - \frac{\pi}{2}} + \frac{1}{2\pi k + \frac{\pi}{2}}| &< \delta \\
|\frac{2\pi k - \frac{\pi}{2} + 2\pi k - \frac{\pi}{2}}{(2\pi k - \frac{\pi}{2})\times(2\pi k + \frac{\pi}{2})}| &< \delta \\
|\frac{4\pi k}{(2\pi k)^2 - (\frac{\pi}{2})^2}| &< \delta
\end{align*}

Fixando agora $k \in \N, k>0$, tem-se:
\begin{align*}
	4\pi k &< \delta\times (2\pi k)^2 - \delta\times(\frac{\pi}{2})^2 \\
	0 &< 4\delta\pi^2 k^2 - 4\pi k - \frac{\delta\pi^2}{4} 
\end{align*}

Considere a \funcao $\f(k) = 4\delta\pi^2 k^2 - 4\pi k - \frac{\delta\pi^2}{4}$. As \raizes da \funcao \sao:
\begin{align*}
	k_1 &= \frac{2 + \sqrt{4 - \delta^2\pi^2}}{4\delta\pi}\\
	k_2 &= \frac{2 - \sqrt{4 - \delta^2\pi^2}}{4\delta\pi}
\end{align*}

Como $\delta < \frac{2}{\pi}$, as duas \raizes \sao reais. Como a \funcao descreve um parabola com abertura para cima, para satisfazer a \inequacao $0 < 4\delta\pi^2 k^2 - 4\pi k - \frac{\delta\pi^2}{4}$, pode-se ter que $k < k_2$ ou que $k > k_1$. Sendo assim, dado $\delta$ qualquer, \eh possivel escolher $k$ de forma que $|x-y| < \delta$ e, alem disso, $|f(x)-f(y)| = |-1 + 2| = 2 > 1 = \epsilon$.

Juntamente ao outro caso, \entao, temos que $\f :(0,+\infty)\rightarrow\R)$ definida por \mbox{$\f (x) = \sin (\frac{1}{x})$} \nao \eh uniformemente \continua.

\textbf{b)}

Usando que $h:\R\rightarrow\R$ dada por $h(x) = \sin(x)$ \eh \diferenciavel em todos os pontos do \dominio e que sua derivada \eh dada por $\cos(x)$. Vimos em aula que \diferenciavel implica \continua.

\Eh notavel que se $\alpha\ge 1$ ou $\alpha \le -1$, \entao $|\sin(\alpha)| \le |\alpha|$, pois $|sin(\alpha)| \le 1$ sempre.

Se $\alpha = 0$, $|sin(\alpha)| = sin(0) = 0 = \alpha \le |\alpha|$.

Tome \entao $\alpha \in (0,1)$. Como $sin(\alpha)$ \eh \continua no intervalo $[0,\alpha]$ e \diferenciavel no intervalo $(0,\alpha)$, \entao $\exists \beta \in (0, \alpha)$, pelo TVM, tal que \mbox{$\sin'(\beta) = \frac{\sin(\alpha) - \sin(0)}{\alpha - 0} \iff \cos(\beta) = \frac{\sin(\alpha)}{\alpha}$}. Novamente usando que $\forall \beta \in \R, \cos(\beta) \leq 1$, $\cos(\beta) = \frac{\sin(\alpha)}{\alpha} \iff  \frac{\sin(\alpha)}{\alpha} \le 1$, mostrando que $sin(\alpha) \le \alpha$ e $|\sin(\alpha)| \le |\alpha|$.

Tome \entao $\alpha \in (-1,0)$. Como $sin(\alpha)$ \eh \continua no intervalo $[\alpha,0]$ e \diferenciavel no intervalo $(\alpha,0)$, \entao $\exists \beta \in (\alpha, 0)$, pelo TVM, tal que \mbox{$\sin'(\beta) = \frac{\sin(0) - \sin(\alpha)}{0-\alpha} \iff \cos(\beta) = \frac{-\sin(\alpha)}{-\alpha}$}. Novamente usando que $\forall \beta \in \R, \cos(\beta) \leq 1$, $\cos(\beta) = \frac{\sin(\alpha)}{\alpha} \iff  \frac{\sin(\alpha)}{\alpha} \le 1$, mostrando que $sin(\alpha) \le \alpha$ e $|\sin(\alpha)| \le |\alpha|$.

Sendo assim, $\forall \alpha \in \R, |sin(\alpha)| \le |\alpha|$.
\\
\hrule
\vspace{1cm}

Considere $a > 0$ qualquer. Dado $\epsilon > 0$ qualquer.

Considere $x,y\in [a,+\infty)$ tais que
$
|g(x) - g(y)| < \epsilon
$, que significa que \mbox{
$
|\sin(\frac{1}{x}) - \sin(\frac{1}{y})| < \epsilon
$}. Usando a indentidade trigonometrica $\sin(w) - \sin(z) = 2\cos(\frac{w+z}{2})\sin(\frac{w-z}{2})$, tem-se que
$
|2\cos(\frac{x+y}{2xy})\sin(\frac{x-y}{2xy})| < \epsilon$ \entao 
$|\cos(\frac{x+y}{2xy})\sin(\frac{x-y}{2xy})| < \frac{\epsilon}{2}
$. \Eh claro que \mbox{$\forall \alpha \in \R, \cos(\alpha) \leq 1$}, portanto, se $|\sin(\frac{x-y}{2xy})| < \frac{\epsilon}{2}$, \entao $|\cos(\frac{x+y}{2xy})\sin(\frac{x-y}{2xy})| < \frac{\epsilon}{2}$. Assim, basta que $\exists \delta, |x-y| < \delta \Rightarrow |\sin\frac{x-y}{2xy}| < \frac{\epsilon}{2}$ para que a \funcao $g$ seja uniformemente \mbox{\continua.} Como mostrado acima, \mbox{$\forall \alpha \in \R, |sin(\alpha)| \le |\alpha|$}, logo, $|\frac{x-y}{2xy}| < \frac{\epsilon}{2} \Rightarrow |\sin\frac{x-y}{2xy}| < \frac{\epsilon}{2}$. 

Assim, basta que $\exists \delta, |x-y| < \delta \Rightarrow |\frac{x-y}{xy}| < \epsilon$. Como $x \ge a$ e $y \ge a$, \entao $|\frac{x-y}{xy}| \le |\frac{x-y}{a^2}|$. Finalmente, basta que:
\begin{align*}
	\exists \delta, |x-y| < \delta &\Rightarrow |\frac{x-y}{a^2}| < \epsilon \\
	&\iff \\
	\exists \delta, |x-y| < \delta &\Rightarrow |x-y| < a^2\epsilon \\
	&\iff \\
	\delta &\le a^2\epsilon
\end{align*}

Sendo assim, tomar $\delta = a^2\epsilon$ garante que $\forall x,y \in [a, +\infty)$, se $|x-y|<\delta$ \entao $|g(x) - g(y)| < \epsilon$.
























	
	\textbf{Q.11}

Note que a \funcao $\f$ \eh limitada, pois $\f(a)$ e $\f(b)$ \sao cotas superiores ou inferiores. Considere a \particao $P_n$ de $[a,b]$ em $n > 0$ intervalos de tamanhos iguais. Isto \eh:
$$
P_n = \{a < a + \frac{b-a}{n} < \ldots < a + (n-1)\frac{b-a}{n} < b\}
$$

Define-se $x_i$ por $x_i = a + i\frac{b-a}{n}$ para $i \in \N\cup {0}, i \leq n$. Tambem se definem $I_i$ da forma $I_i = [x_{i-1}, x_i]$ para $i \in \N, i \leq n$.


Subtraindo a soma inferior da supereior:

\begin{align*}
	S(\f,P_n) - s(\f,P_n) &= \sum_{i=1}^{n} \sup(\f(I_i))\times|I_i| - \sum_{i=1}^{n} \inf(\f(I_i))\times|I_i|
\end{align*}

Considerando $\f$ decrescente, $\sup(\f(I_i)) = f(x_{i-1})$ e $\inf(\f(I_i)) = f(x_i)$. Assim, 

\begin{align*}
	S(\f,P_n) - s(\f,P_n) &= \sum_{i=1}^{n} \f (x_{i-1})\frac{b-a}{n} - \sum_{i=1}^{n} \f (x_i)\frac{b-a}{n} \\
	&= \frac{b-a}{n}(\sum_{i=1}^{n}(\f (x_{i-1}) - \f (x_i))) \\
	&= \frac{b-a}{n}(\f (x_0) - \f (x_1) + \f (x_1) - \f (x_2) + \ldots + \f(x_{n-1}) - \f(x_n)) \\
	&= \frac{b-a}{n}(f(x_0) - f(x_n)) \\
	&= \frac{b-a}{n}(f(a)-f(b)) \\
\end{align*}

Tome $\epsilon > 0$ qualquer. Tomando $n \in \N$, o tamanho da \particao, tal que \mbox{$n > \frac{b-a}{\epsilon}(f(a)-f(b))$}, \entao a \subtracao da soma inderior da supereior sera menor que $\epsilon$, visto:

$$
S(\f,P_n) - s(\f,P_n) = \frac{b-a}{n}(f(a)-f(b)) < \frac{b-a}{\frac{b-a}{\epsilon}(f(a)-f(b))}(f(a)-f(b)) = \epsilon
$$

Sendo assim, toda \funcao monotona descrescente \eh integravel, pelo \criterio de integrabilidade visto em aula.

Considerando $\f$ crescente, $\sup(\f(I_i)) = f(x_i)$ e $\inf(\f(I_i)) = f(x_{i-1})$. Assim, 

\begin{align*}
	S(\f,P_n) - s(\f,P_n) &= \sum_{i=1}^{n} \f (x_i)\frac{b-a}{n} - \sum_{i=1}^{n} \f (x_{i-1})\frac{b-a}{n} \\
	&= \frac{b-a}{n}(\sum_{i=1}^{n}(\f (x_i) - \f (x_{i-1}))) \\
	&= \frac{b-a}{n}(-\f (x_0) + \f (x_1) - \f (x_1) + \f (x_2) - \ldots - \f(x_{n-1}) + \f(x_n)) \\
	&= \frac{b-a}{n}(f(x_n) - f(x_0)) \\
	&= \frac{b-a}{n}(f(b)-f(a)) \\
\end{align*}

Tome $\epsilon > 0$ qualquer. Tomando $n \in \N$, o tamanho da \particao, tal que \mbox{$n > \frac{b-a}{\epsilon}(f(b)-f(a))$}, \entao a \subtracao da soma inderior da supereior sera menor que $\epsilon$, visto:

$$
S(\f,P_n) - s(\f,P_n) = \frac{b-a}{n}(f(b)-f(a)) < \frac{b-a}{\frac{b-a}{\epsilon}(f(b)-f(a))}(f(b)-f(a)) = \epsilon
$$

Sendo assim, toda \funcao monotona crescente \eh integravel, pelo \criterio de integrabilidade visto em aula.



	
	\textbf{Q.12}

Tome $a<b\in I$. Tome $x\in [a,b]$. 

\begin{align*}
	x &= \frac{b-a}{b-a}x \\
	&= \frac{-ax + bx}{b-a} \\
	&= \frac{ab -ax + bx - ab}{b-a} \\
	&= \frac{a(b - x) + b(x - a)}{b-a} \\
	&\overset{t = \frac{b-x}{b-a}}{=} ta + \frac{b-b+x-a}{b-a}b \\
	&= ta + (\frac{b-a}{b-a} - \frac{b-x}{b-a})b \\
	&= ta + (1-t)b
\end{align*}

Note que $\forall x \in [a,b], t = \frac{b-x}{b-a}$ \eh tal que $t \in [0,1]$.

Caso $x = a$, $t=1$ e $f(ta + (1-t)b) = f(ta) = f(a) = 1f(a) = tf(a) + (1-t)f(b)$. 

Caso $x = b$, $t=0$ e $f(ta + (1-t)b) = f(b) = 1f(b) = tf(a) + (1-t)f(b)$. 

Caso $a < x < b$, temos $f(x) \leq f(a) + \frac{f(b)- f(a)}{b-a}(x-a)$. Tomando, como acima visto, $t = \frac{b-x}{b-a}$, temos $f(x) = f(ta + (1-t)b)$. Assim:

\begin{align*}
	f(ta + (1-t)b) &\leq f(a) + \frac{f(b)- f(a)}{b-a}(x-a) \\ 
	&= f(a) + (f(b) - f(a))\frac{(b - a) - (b-x)}{b-a} \\
	&= f(a) + (f(b) - f(a))(1-t) \\
	&= f(a) - f(a) + tf(a) + f(b) - tf(b) \\
	&= tf(a) + (1-t)f(b)
\end{align*}

Assim temos que dados $a < b \in I$, qualquer $x \in [a,b]$ pode ser escrito como $ta + (1-t)b$ com $t\in [0,1]$ e que $f(ta + (1-t)b) \leq tf(a) + (1-t)f(b)$, como desejado.








	
	\textbf{Q.13}

Como visto em 12, $\forall x \in [a,b], \exists t \in [0,1]; f(x)\le tf(a) + (1-t)f(b)$. Isso mostra que a \funcao \eh limitada superiorment e  $\forall x \in [a,b], f(x) \le \max(f(a), f(b))$.

Tome $c \in (\frac{b-a}{2},b)$. Como $d$ convexa em $[a,c], \frac{b-a}{2} \in [a,c]$,
$$
\frac{b-a}{2} = (1-t)a + tc
$$
Claro que $t = \frac{b-a}{2(c-a)}$, pois:
\vspace{-0.5cm}
\begin{align*}
	(1-t)a + tc &= \frac{2ca - 2a^2 -ba + a^2 + bc - 	ac}{2(c-a)}\\
	&=\frac{ca - a^2 -ba + bc}{2(c-a)}\\
	&=\frac{(c-a)(a + b)}{2(c-a)}\\
	&= \frac{a+b}{2}
\end{align*}

Logo,
\begin{align*}
	f(\frac{a-b}{2}) &\leq tf(c) + (1-t)f(a)\\
	&= \frac{b-a}{2(c-a)}f(c) + \frac{2(c-a) - (b-a)}{2(c-a)}f(a)
\end{align*}

Isolando $f(c)$, tem-se:
\begin{align*}
	f(c) &\ge \frac{f(\frac{b-a}{2}) - \frac{2(c-a) - (b-a)}{2(c-a)}f(a)}{\frac{b-a}{2(c-a)}} \\
	&= \frac{2(c-a)}{b-a}f(\frac{b-a}{2}) - \frac{2(c-a) - (b-a)}{b-a}f(a)\\
	&= \frac{2(c-a)}{b-a}f(\frac{b-a}{2}) - \frac{2c -a - b}{b-a}f(a)\\
	&\ge -2|f(\frac{b-a}{2})| - \frac{2c -a - b}{b-a}f(a)\\
	&\ge -2|f(\frac{b-a}{2})| - |\frac{2c -a - b}{b-a}||f(a)|\\
	&\ge -2|f(\frac{b-a}{2})| - |\frac{b -a}{b-a}||f(a)| \\
	&= -2|f(\frac{b-a}{2})| - |f(a)|
\end{align*}

Sendo assim, $\forall x\in[\frac{b+a}{2}, b], f(x)$ \eh limitada inferiormente \tambem. Pode-se usar processo analogo para mostrar que isso \tambem vale para $x\in[a,\frac{b+a}{2}]$. Isso mostra que $f(x)$ \eh limitada. Assim, $\exists L \ge |f(x)|; x\in [a,b]$.

Tomando agora $x < y \in [a,b]$, temos:
$x = (1-t)a + ty$, com $t = \frac{x-a}{y-a}$. Tem-se
\begin{align*}
	f(x) - f(y) &\leq (1-t)f(a) + tf(y) - f(y) \\
	&= (1-t)(f(a) - f(y))\\
	&= \frac{y-x}{y-a}(f(a) - f(y))\\
	&\leq \frac{2L(y-x)}{y-a}
\end{align*}

Dado $\epsilon > 0$. Tome $y \in (a,b)$ e seja $\beta = y-a$. Note que se fixarmos $\delta = \frac{\epsilon\beta}{4L}$, \entao $|y-x|<\delta \Rightarrow |f(y)-f(x)|<\epsilon$, pois $|f(x)-f(y)| \leq |\frac{2L(y-x)}{y-a}| \leq |\frac{2L\delta}{\beta}|$ e, pela escolha de $\delta$, $|\frac{2L\delta}{\beta}| = |\frac{2L\epsilon\beta}{4L\beta}| = \frac{\epsilon}{2}$. Sendo assim, toda \funcao convexa em $I = [a,b]$ \eh \continua em $(a,b)$.

\vspace{1cm}
\hrule
\vspace{1cm}

A \funcao $f:[0,1]\rightarrow\R$ dada por
$f(x) = \begin{cases*}
	0, \text{se } x\in (0,1) \\
	1, \text{cc}
\end{cases*}$ \eh convexa em $[0,1]$, mas \nao \eh continua nos pontos $x=0$ e $x=1$. Tome $x,y,t \in [0,1]$ e considere $f(tx + (1-t)y)$. Sem perda de generalidade, suponha $x \le y$.

Se $x = y$, $f(tx + (1-t)y) = f(x) = tf(x) + (1-t)f(x) = tf(x) + (1-t)f(y)$.

Se $tx + (1-t)y \in (0,1)$, $f(tx + (1-t)y) = 0$. Como $\forall x \in [0,1], f(x) \ge 0$, \entao $f(tx + (1-t)y) \le tf(x) + (1-t)f(y)$.

Se $tx + (1-t)y = 0$, \entao $t = 1, x = 0$, caso contrario, $1-t \neq 0, y \neq 0 \Rightarrow (1-t)y > 0 \Rightarrow tx + (1-t)y>0$. Assim, $f(tx + (1-t)y) = f(0) = 1\times f(x) + 0 \times f(y) = tf(x) + (1-t)f(y)$.


Se $tx + (1-t)y = 1$, \entao $t = 0, y = 1$. $f(tx + (1-t)y) = f(1) = 0\times f(x) + 1 \times f(1) = tf(x) + (1-t)f(y)$.

Sendo assim, ela \eh convexa.

Considere a \sequencia \seqn{x} com $x_n = \frac{1}{n}$. Ja foi visto que \limi{x_n}{0}. No entanto, pela \definicao da $f$, $\forall n \in \N, f(x_n) = 0$ e, por isso, \limi{f(x_n)}{0} e $0\neq 1 = f(0)$.

Considere, agora, a \sequencia \seqn{x} com $x_n = \frac{n-1}{n}$. Ja foi visto que \limi{x_n}{1}. No entanto, pela \definicao da $f$, $\forall n \in \N, f(x_n) = 0$ e, por isso, \limi{f(x_n)}{0} e $0\neq 1 = f(0)$.

Sendo assim, $f$ \nao \eh \continua nem em $x=0$ nem em $x=1$.






	
	\textbf{Q.14}

\textbf{a)}

Usando a metrica habitual $d(x, y) = |y-x|, \forall x,y \in \R$. Assim, usarei que \mbox{$|f(y)-f(x)| \le K|y-x|$}. O valor $K$ pode ser qualquer constante lipshitz de $f$.

Dada uma \particao $P = \{a=t_0 < t_1< \ldots < t_n = b\}$ qualquer. A \variacao \eh \mbox{$V(f;P) = \sum_{i = 1}^n|f(t_i) - f(t_{i-1})|$}. Como \mbox{$|f(y)-f(x)| \le K|y-x|$}, \entao tem-se que $\sum_{i = 1}^n|f(t_i) - f(t_{i-1})| \leq K\times\sum_{i = 1}^n|t_i - t_{i-1}| = K\times(|t_1 - t_0| + |t_2 -t_1| + \ldots + |t_n - t_{n-1}|)$. Note que $|t_n - t_0| = |t_1 - t_0| + |t_2 -t_1| + \ldots + |t_n - t_{n-1}|$.

Temos \entao, $V(f;P) \leq K\times (t_n - t_0) = K(b-a)$. Isso mostra que $\forall P$ \particao de $[a,b]$, $\underset{P}{sup}\{V(f;P); P \text{ \particao de }[a,b]\}$ \eh finito.

\textbf{b)}

Dada uma \particao $P = \{a=t_0 < t_1< \ldots < t_n = b\}$ qualquer. Como $f$ monotona, \entao $f$ limitada, pois $f(a),f(b)$ \sao cotas superiores ou inferiores.

Supondo que $f$ seja crescente.

$V(f;P) = \sum_{i = 1}^n|f(t_i) - f(t_{i-1})| = \sum_{i = 1}^n(f(t_i) - f(t_{i-1}))$, pois $\forall i, 1 \le i \le n$ tem-se $f(t_i) \ge f(t_{i-1})$. Sendo assim,
$$V(f;P) = f(t_1) - f(t_0) + f(t_2) - f(t_1) + \ldots + f(t_n) - f(t_{n-1}) = f(t_n) - f(t_0)<\infty$$
Supondo que $f$ seja decrescente.

$V(f;P) = \sum_{i = 1}^n|f(t_i) - f(t_{i-1})| = \sum_{i = 1}^n(f(t_{i-1}) - f(t_i))$, pois $\forall i, 1 \le i \le n$ tem-se $f(t_{i-1} \ge f(t_i))$. Sendo assim,
$$V(f;P) = f(t_0) - f(t_1) + f(t_1) - f(t_2) + \ldots + f(t_{n-1}) - f(t_n) = f(t_0) - f(t_n)<\infty$$
Temos \entao que tanto monotonas crescentes quanto decrescentes tem \variacao limitada.

\textbf{c)}

Considere $f:[a,b]\rightarrow\R, g:[a,b]\rightarrow\R, h:[a,b]\rightarrow\R$ com $h(x) = f(x) + g(x)$. Tome $V(h;P) = \sum_{i = 1}^n|h(t_i) - h(t_{i-1})| = \sum_{i = 1}^n|f(t_i) + g(t_i) - f(t_{i-1}) - g(t_{i-1})|$. Pela desigualdade triangular, provada anteriormente, 
$$\sum_{i = 1}^n|[f(t_i) - f(t_{i-1})] + [g(t_i) - g(t_{i-1})]| \leq
\sum_{i = 1}^n(|f(t_i) - f(t_{i-1})| + |g(t_i) - g(t_{i-1})|)$$

Note \mbox{$\sum_{i = 1}^n(|f(t_i) - f(t_{i-1})| + |g(t_i) - g(t_{i-1})|) = \sum_{i = 1}^n|f(t_i) - f(t_{i-1})| + \sum_{i = 1}^n |g(t_i) - g(t_{i-1})|$}. Sendo assim, $V(h;P) \leq \sum_{i = 1}^n|f(t_i) - f(t_{i-1})| + \sum_{i = 1}^n |g(t_i) - g(t_{i-1})| < \infty$, pois tanto $\sum_{i = 1}^n|f(t_i) - f(t_{i-1})| < \infty$ quanto $\sum_{i = 1}^n |g(t_i) - g(t_{i-1})| < \infty$.

Considere $f:[a,b]\rightarrow\R, g:[a,b]\rightarrow\R, h:[a,b]\rightarrow\R$ com $h(x) = f(x)g(x)$. Considere:
\begin{align*}
	V(h;P) &= \sum_{i = 1}^n|h(t_i) - h(t_{i-1})| \\
	&= \sum_{i = 1}^n|f(t_i)g(t_i) - f(t_{i-1})g(t_{i-1})| \\
	&= \sum_{i = 1}^n|f(t_i)g(t_i) - f(t_i)g(t_{i-1}) + f(t_i)g(t_{i-1}) - f(t_{i-1})g(t_{i-1})| \\
	&= \sum_{i = 1}^n|f(t_i)(g(t_i) - g(t_{-1})) + g(t_{i-1})(f(t_i) - f(t_{i-1}))| \\
	&\leq \sum_{i = 1}^n(|f(t_i)(g(t_i) - g(t_{-1}))| + |g(t_{i-1})(f(t_i) - f(t_{i-1}))|) \\
	&= \sum_{i = 1}^n(|f(t_i)||(g(t_i) - g(t_{-1}))| + |g(t_{i-1})||(f(t_i) - f(t_{i-1}))|)
\end{align*}

Usando que \variacao limitada em $[a,b]$ implica \funcao limitada.

\vspace{1cm}
\hrule
\vspace{1cm}

Prova:

Tomando $x \in [a,b]$ e uma \particao $P$ de $[a,b]$ com $P = \{x_0 < \ldots < n_n\}$, tem-se
\begin{align*}
	|f(x)| &= |f(x)| + |f(a)| - |f(a)| \\
	&\leq |f(a)| + |f(x) - f(a)| \text{ (desigualdade triangular da \subtracao)} \\
	&\leq |f(a)| + V_a^b(f) \\
	&< \infty
\end{align*}
Se $x \in (a,b)$, a penultima desigualdade \eh valida pois $|f(x) - f(a)| \leq \sup_P\{V(f;P) | P \text{ \particao de } [a,b]\}$,
ja que pode-se tomar \particao $P = \{a < x <b\}$ e $V(f;P) = |f(b) - f(x)| + |f(x) - f(a)| \ge |f(x) - f(a)|$. Note que se $x = a$ ou $x=b$ a resposta \eh igualmente simples. Basta tomar a \particao $P=\{a < x_0 < b\}, x_0 \in (a,b)$ e notar que $|f(x) - f(a)| \leq |f(b) - f(x_0)| + |f(x_0) - f(a)|$, independentemente do caso.

\vspace{1cm}
\hrule
\vspace{1cm}


Como $f$ limitada, $\exists L$ tal que $\forall i \in \N, i \leq n$ tem-se $f(t_i) < L$ e $g(t_i) < L$. Seguindo \entao que:
\begin{align*}
	V(h;P) &\leq \sum_{i = 1}^n(|f(t_i)||(g(t_i) - g(t_{-1}))| + |g(t_{i-1})||(f(t_i) - f(t_{i-1}))|) \\
	&< L\times ( \sum_{i = 1}^n|g(t_i) - g(t_{-1})| + \sum_{i = 1}^n|f(t_i) - f(t_{-1})|)
\end{align*}
Da mesma forma, como $f,g$ \sao de \variacao limitada, $\exists M$ tal que $ \sum_{i = 1}^n|g(t_i) - g(t_{-1})| < M$ e $\sum_{i = 1}^n|f(t_i) - f(t_{-1})| < M$, segue,finalmente que:

$$
V(h;P) < 2\times LM < \infty
$$

Sendo assim, $h = fg$ \eh de \variacao limitada. 

Tome, agora $h=|f|$ e alguma \particao $P$ de $[a,b]$.

\begin{align*}
		V(h;P) &= \sum_{i = 1}^n|h(t_i) - h(t_{i-1})| \\
		&= \sum_{i = 1}^n||f(t_i)| - |f(t_{i-1})|| \\
		&\leq \sum_{i = 1}^n|f(t_i) - f(t_{i-1})| \\
		&< \infty
\end{align*}

Note que $||f(t_i)| - |f(t_{i-1})|| \leq |f(t_i) - f(t_{i-1})| $ e isso \eh a desigualdade triangular da \subtracao que ja foi provada anteriormente. Dessa forma, $h = |f|$ \eh de \variacao limitada. 








	
	\textbf{Q.15}

Temos 
\begin{align*}
	V^a_b (f) &= \underset{P}{\sup}\{V(f;P); P \text{ \eh \particao de } [a,b]\} \\
	&= \underset{P}{\sup}[\sum_{i=1}^n |f(t_i) - f(t_{i-1})|; \text{ com } P = \{t_0 < \ldots < t_n\}]
\end{align*}

Temos \tambem 
\begin{align*}
	\int_a^b|f'(x)|dx &= \underset{P}{\sup}\{s(|f'|;P); P \text{ \eh \particao de } [a,b]\} \\
	&= \underset{P}{\inf}\{S(|f'|;P); P \text{ \eh \particao de } [a,b]\} \\
	&= \underset{P}{\sup}\{\sum_{i=1}^n \inf (|f'([t_{i-1}, t_i])|)\cdot(t_i - t_{i-1}); \text{ com } P = \{t_0 < \ldots < t_n\}\} \\
	&= \underset{P}{\inf}\{\sum_{i=1}^n \sup (|f'([t_{i-1}, t_i])|)\cdot(t_i - t_{i-1}); \text{ com } P = \{t_0 < \ldots < t_n\}\}
\end{align*}

Temos ainda, com $A \subset \R, a \in A\cap A'$, $x\in A, g:A\rightarrow\R$:
\begin{align*}
	g'(a) &= \lim_{x\rightarrow a} \frac{g(x)-g(a)}{x-a} \\
	g'([A]) &= \{g'(a); a\in A\} \\
	&= \{\lim_{x\rightarrow a} \frac{g(x)-g(a)}{x-a}; a \in A\}
\end{align*}

Fixando agora uma \particao qualquer $P = \{a=x_0 < \ldots < x_n = b\}$. Tem-se que:
\begin{align*}
	|f(x_i) - f(x_{i-1})| &= |\int_{x_{i-1}}^{x_i} f'(t)dt| \text{ (pois $f$ \continua -TFC}) \\
	&\leq \int_{x_{i-1}}^{x_i} |f'(t)|dt \text{ (soma de modulos)}
\end{align*}

Isso mostra que:
\begin{align*}
	\sum_{i=1}^n |f(t_i) - f(t_{i-1})| &\leq \sum_{i=1}^n \int_{x_{i-1}}^{x_i}|f'(t)|dt \\
	&= \int_{a}^{b}|f'(t)|dt
\end{align*}

Como $P$ qualquer, temos $V_a^b (f) \leq \int_{a}^{b}|f'(t)|dt$.

Como $f'$ \eh \continua, dado $\epsilon$ qualquer, $\exists \delta$ tal que $x,y \in [a,b], |x-y| < \delta \Rightarrow |f'(x) - f'(y)| < \epsilon$.

Fixando $\epsilon$, tome $\delta$ tal que $x,y \in [a,b], |x-y| < \delta \Rightarrow |f'(x) - f'(y)| < \epsilon$. Tome \tambem a \particao $P$ que separa em $n = \ceil{\frac{b-a}{\delta}}$ intervalos iguais. Isso garante que $P = \{a=x_0,\ldots,x_n=b\}$ \eh tal que todo \mbox{$i\in\N,i<n;|x_i - x_{i-1}| < \delta \Rightarrow x_{i-1} \leq x \leq x_i$}; $|f'(x_i) - f'(x)| < \epsilon$. Como $|f'(x_i) - f'(x)| \ge |f'(x_i)| - |f'(x)|$, temos \tambem que $|f'(x)| \le \epsilon + |f'(x_i)|$.

Assim,
\begin{align*}
	\int_{x_{i-1}}^{x_i} |f'(t)|dt &\le |x_i-x_{i-1}|(|f'(x_i)| + \epsilon) \\
	&= |\int_{x_{i-1}}^{x_i} f'(x_i)dt| + |x_i-x_{i-1}|\epsilon \\
	&= |\int_{x_{i-1}}^{x_i} f'(t) + f'(x_i) - f'(t)dt| + |x_i-x_{i-1}|\epsilon \\
	&\le |\int_{x_{i-1}}^{x_i} f'(t) dt|  + |\int_{x_{i-1}}^{x_i} f'(x_i) - f'(t)dt| + |x_i-x_{i-1}|\epsilon \\
	&\le |\int_{x_{i-1}}^{x_i} f'(t) dt|  + |x_i-x_{i-1}|\epsilon + |x_i-x_{i-1}|\epsilon \\
	&= |f(x_i) - f(x_{i-1})| + 2|x_i-x_{i-1}|\epsilon
\end{align*}

Portanto, tem-se que:
\begin{align*}
	\int_{a}^{b} |f'(t)|dt &\le \sum_{i=1}^{n} (|f(x_i) - f(x_{i-1})| + 2|x_i-x_{i-1}|\epsilon) \\
	&= 2n\epsilon + \sum_{i=1}^{n} |f(x_i) - f(x_{i-1})| \\
	&= 2n\epsilon + V_a^b (f) \\
\end{align*}

Como o $\epsilon$ \eh arbitrario, para qualquer valor $\alpha > 0$, $	\int_{a}^{b} |f'(t)|dt \le \alpha + V_a^b (f)$ e isso mostra que $\int_{a}^{b} |f'(t)|dt \le V_a^b (f)$.

Sendo assim, temos finalmente que $\int_{a}^{b} |f'(t)|dt = V_a^b (f)$ para $f \in C^1([a,b])$.






%Sendo assim, 
%\begin{align*}
%	|f'([t_{i-1}, t_i])| &= \{ | \lim_{x\rightarrow a} \frac{f(x)-f(a)}{x-a}|; x,a \in [t_{i-1}, t_i]\} \\
%	| \lim_{x\rightarrow a} \frac{f(x)-f(a)}{x-a}| &\ge \frac{| \lim_{x\rightarrow a} f(x)-f(a)|}{t_i - t_{i-1}}
%\end{align*}
%
%Logo, com $x,a \in [t_{i-1}, t_i]$ em cada inf:
%\begin{align*}
%	\int_a^b|f'(x)|dx &= \underset{P}{\sup}\{\sum_{i=1}^n \inf (| \lim_{x\rightarrow a} \frac{f(x)-f(a)}{x-a}|)\cdot(t_i - t_{i-1}); \text{ com } P = \{t_0 < \ldots < t_n\}\} \\
%	&\ge \underset{P}{\sup}\{\sum_{i=1}^n \inf (\frac{| \lim_{x\rightarrow a} f(x)-f(a)|}{t_i - t_{i-1}})\cdot(t_i - t_{i-1}); \text{ com } P = \{t_0 < \ldots < t_n\}\} \\
%	&= \underset{P}{\sup}\{\sum_{i=1}^n \inf (| \lim_{x\rightarrow a} f(x)-f(a)|); \text{ com } P = \{t_0 < \ldots < t_n\}\} \\
%\end{align*}






	
\end{document}














